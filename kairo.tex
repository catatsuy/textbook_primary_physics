 \section{コンデンサー}

 導体間に電荷を移動させ保持したとき,これらの導体は{\bfseries コンデンサーを形成する}という。
 このとき移動させた電荷の大きさ$Q$\kyakuchuu{導体間の帯びている電荷の差ではない}をコンデンサーに蓄えられた電荷という。
 また$C=\bunsuu{Q}{V}$を(電気)容量と\textgt{\bfseries 定義}する。また容量の単位は\unit{\ruby{F}{\text{ファラド}}}である。

 よってコンデンサーは回路だけではなく例えば,下敷きで髪をこすっても電荷移動は起こるからコンデンサーであるし,
 雷雲と地面も電荷移動が起こっているからコンデンサーである。

  \subsection{電気容量は定数か?}

  Coulombの法則と重ね合わせの原理より\textgt{\bfseries 金属表面の電荷分布を変えず}に$Q$を2倍にすると,
  電場・電位ともに2倍となる。よって電気容量は極板の形・配置・極板間の物質分布による
  定数とみなせる\textgt{\bfseries 可能性がある}。

  しかし雷雲と地面のように電荷がたまりすぎると放電が起きることがあるので
  コンデンサーで蓄えられる電荷には限界がある。よって実際にはある程度の範囲で定数であることが
  知られており,その証明は大学でやる\kyakuchuu{電気容量が定数であることの証明はまだやっていないことを念頭に入れて欲しい}。

  \subsection{コンデンサーの静電エネルギー}

  静電エネルギーとは『考えている電荷分布を作るのに
  必要な静電気力につりあわせる外力のする仕事』であり,これはもちろんコンデンサーのときにも当てはまる。

  よって電荷に偏りのない状態から電荷を移動させてコンデンサーと同じ電荷分布を作るわけだが,
  一気に電荷$Q$を動かすと電場が複雑に変化し計算できない。
  よって静電気力が保存力であることを利用して計算する方法を考えよう。

  ここで物理ではおなじみの考え方。電位$v$の変化が無視できるくらいの
  微小な電荷$dq$を移動させるのに必要な外力を$q=0\text 〜Q$まで足せば静電エネルギーは出るはずである。

  よって求める外力の大きさは電位の定義から$v\,dq=\bunsuu{q}{C}\,dq$だから$Q=CV$とすると

  $$\bm{U=\dint{0}{Q}\bunsuu{q}{C}\,dq=\bunsuu{Q^2}{2C}=\bunsuu12QV=\bunsuu12CV^2}$$

  となる。これは既知として使ってよい。

  \section{(無限に広いとみなせる)平行金属板コンデンサー}




  \subsection{極板間への物体挿入(導体)}




  \subsection{極板間への物体挿入(誘電体)}




 \section{電流}
 %\subsection{電流の定義}
 %
 大きさは断面を単位時間当たり\kyakuchuu{無論,単位時間中に通る電流が一定である保証は無い}に通る電気量。
 よって単位は\unit{\ruby{A}{\text{アンペア}}}${}={}$\unit{C/s}である。

 そして正方向を正電荷の流れと同じ向き
 \kyakuchuu{つまり負電荷である電子の流れとは{\bfseries 逆向き}}とする。
 これは導線内の電場の向きと同じである。

 ただし電場が生じているなら自由電子はどんどん加速されて速さが大きくなり電流は大きくなるはずである。
 しかし実験的に電流は定常状態になることが分かっているので,熱振動している陽イオンなどに自由電子がぶつかり減速もされる
 と考えられる\kyakuchuu{ただし,\ref{抵抗?}も読んで欲しい}。

 この二つの原因から電流は平衡状態となり,平均移動速度$\overline{v}$は一定になる。

 %図

 $I=enS\overline{v}$であることが容易に分かる。

 \section{電気抵抗の定義と特徴}

 電気抵抗を$R=\bunsuu{V}{I}$と定義する(単位は\unit{\ruby{$\Omega$}{\text{オーム}}}${}={}$\unit{V/A})。

 この電気抵抗は電場$E$の大きさが2倍になれば,電流$I$も自由電子の加速度が2倍になるのだから2倍になると考えられ,
 また電圧(電位と同じだが特に電気回路では電圧と呼ぶことが多い)$V$も2倍になると考えられる。

 よって電気抵抗$R$は一定であると考えられ,またそれは実験的にも正しいことが分かっている。
 また$R$は
 \begin{enumerate}
  \item 物質の種類(主に$n$の違い)
  \item 金属の温度(陽イオンの熱振動が自由電子の衝突に影響していると考えられる)
  \item 長さ$l$,断面積$S$($S$が違うと$I$は変化し,$I$が同じ時は$E$が同じなので,$l$が違うと$V$は異なる)
 \end{enumerate}
 に影響されると考えられる。

 \ajMaru{1},\ajMaru{2}を固定したときの,$R$の\ajMaru{3}の依存性

 $S$を2倍にする。

 %図

 $l$を2倍にする。

 %図

 よって$R\propto\bunsuu{l}{S}$と考えられ,$R=\rho\bunsuu{l}{S}$をかくとき,$\rho\unit{\Omega\cdot m}$を抵抗率という。
 そして抵抗率$\rho$が\ajMaru{1},\ajMaru{2}に依存すると考えられる。

 \begin{description}
  \item[$\bm\rho$の\ajMaru{1}による依存性]$n$の大きい金属の方が$\rho$は小さい傾向にある。
  \item[$\bm\rho$の\ajMaru{2}による依存性]金属では,温度により自由に動けない電子が自由に動けるようになったりは
	     ほとんどしない\kyakuchuu{半導体では温度の上昇と共に$n$が大きくなり,この影響により$\rho$は小さくなる傾向がある}。
	     よって$n$は温度にはほとんど影響を受けない。しかし温度があがると\kenten{\bfseries 陽イオンなどの
	     熱振動が激しくなり},自由電子の散乱が起こりやすくなり,$\rho$は大きくなる傾向にある。
 \end{description}

  \subsection{しかし,圏点部分をよく考えてみよう!}\label{抵抗?}

  実はこれは大嘘である。

  randomな振動をしている陽イオンは一定間隔で分布していて,randomな振動をしているのだから,
  陽イオン同士には近づいたり,遠ざかったりするものがrandomにある。よって電子から見た平均の格子間隔は変わらないはず
  (おかしいと気づかないならもう一度見直してよく考えよう)。そうするとぶつかりやすさも変わらないはずである。古典物理学的に
  言うなら,自由電子の振動が激しくなってすぐに衝突してしまうという方が適切であるが,陽イオンの熱振動で議論するのが普通である。

  さらにいうと実験的には絶対零度に近い温度では抵抗が0になる(超伝導現象)が起こることが分かっている。
  よって単純に電子が陽イオンに『\textgt{\bfseries ぶつかる}』というイメージは\textgt{\bfseries 正確でない}。

  実際,この抵抗の問題は量子力学がない時代には超難問とされてきた。
  そして現在これは量子力学でないと説明不可能であることが知られていて,
  その量子力学によると抵抗の原因は結晶格子の乱雑さ(規則正しければ0)によるものとされている。

  ちなみに半導体は温度が上がると動ける自由電子が増加するので$n$が増加する。よって温度が上がると抵抗率は減少する。

  \subsection{Joule熱について}

  単位時間あたりに導線内の自由電子は,電場から$IV$の仕事をされるが,これを陽イオンなどとの衝突により
  それらの熱振動を激しくするのに使われる\kyakuchuu{$IV$の仕事をもらっているのに平均移動速度は不変なのはこういう理由からである}。

  よって単位時間あたりに失われるenergyは$P=IV=RI^2=V^2/R\unit{J/s}$

  \section{(準定常電流)回路の考え方}

  準定常電流回路とは電磁場の発生が無視でき,それにより回路から空間に持ち出されるenergyが無視できる回路のこと。
  そのように仮定すれば電荷電流の分布を楽に求めることができる。

  \subsection{回路を解く方法}
  \begin{itemize}
   \item 電荷・電流を仮定(もちろん物理法則に反しないように)
   \item 以下の回路方程式で求める
	 \begin{itemize}
	  \item 孤立系の電荷保存
	  \item 任意のループ1周に対し,(ループ1周あたりの起電力)${}={}$(ループ1周あたりの電圧降下)
	 \end{itemize}
  \end{itemize}

  ただし起電力とは電荷に対して仕事を\textgt{\bfseries 恒常的}\kyakuchuu{ここが電位差と違う}にする能力のことである。
  また電圧とは電位差のことで特に回路で大変よく使われる。
  その理由は『ぐるぐる』電場上のある点Aを考えると『ぐるぐる』電場を電荷が点Aから1周して点Aに戻ってきたときに
  電荷は仕事をもらっている。よってこのとき電荷に働く力は保存力ではない。よって電位が定義できないので電位差が使われる。%??
  %過度現象とか例題を載せたほうが良いのか?

  \subsection{energy収支}

  運動方程式の両辺にスカラー積で速度$v$をかけると仕事率の式に変わる。実は回路方程式にも似たようなことができる。

  両辺に電流$i$をかけると\kyakuchuu{もちろん回路方程式はスカラー}仕事率の式にかわる。
  それを時刻$t=0\text 〜\infty$において積分すれば仕事になることは明らかである。

  \section{電磁場中での荷電粒子の運動}
  荷電粒子$q$に電場$\bm E$,磁束密度$\bm B$の元で,実験的に以下のような力$\bm F$が及ぼされることが分かっている。
  $$\bm F=q(\bm E+\bm v\times\bm B)$$
  この力はLorentz力と呼ばれる。またベクトル積の特徴から$\bm v\times\bm B$の項は$\bm v$と直行しているので,
  電荷に対して{\bfseries 仕事をしない}。また$\bm B$方向には磁場から電荷が受ける力は働かない。

  よって荷電粒子の運動を求めるにはLorentz力と運動方程式から解けばよいが電場や磁場が時間的に変化すると
  複雑な積分計算となるから,特に高校範囲では一様電場・磁場になる。

  \subsection{電磁場中での電荷と磁荷の運動}

  ここで電荷,磁荷が電磁場から受ける力をまとめると
  \begin{gather*}
   \bm F=q(\bm E+\bm v\times\bm B)\\
   \bm F=m(\bm H-\bm v\times\bm D)
  \end{gather*}
  と表せる。ただし$M$は質量ではなくmagnetic(磁荷)のことである。

  ここで$H,D$はそれぞれ“磁場”,“電束密度”と呼ばれ,それぞれ電場,磁束密度に対応する値である。

  電場・電束密度,磁場・磁束密度は単なる測り方の違いでそれぞれ真空中では$\epsilon_0 E=D, \mu_0 H=B$の関係式が成立している。

  また$\epsilon_0, \mu_0$は真空の誘電率,真空の透磁率と呼ばれる定数である。

  この式を見ればなぜ電気と磁気が同じ“電磁”気学となったのかが少しわかるだろう%
  \kyakuchuu{そうなるとなぜ磁荷が無いのか大変疑問である}。

 \section{電流が磁場から受ける力}

 歴史的には電流が磁場から受ける力が最初に観測され,それについて研究をしてLorentz力は発見されたが,
 ここではLorentz力を前提にして話をしてみたい。

 $\bm I=-enS\bm v$とかける\kyakuchuu{$\bm I$と$\bm v$は逆向きである}。
 よってこの導線の長さ$l$の部分には,$nSl$個自由電子が含まれることに注意して,
 各自由電子が$\bm f=-e\bm v\times\bm B$のLorentz力を受けているので,
 この部分が磁場から受ける力は,
 \begin{align*}
  F&=nSl\bm f=nSl(-e\bm v\times\bm B)\\
  {}&=l(-enS\bm v)\times\bm B=l\bm I\times\bm B
 \end{align*}

 また磁束密度$\bm B$の単位は右辺と左辺を比較して\unit{\ruby{T}{テスラ}}${}={}$\unit{N/A\cdot m}である。


 \section{磁場のGaussの法則}

 今度は電磁場の法則\ajMaru 2を定式化しよう。\ref{gauss}でやったことと全く同じである。
 しかし,単磁荷が発見されてないので磁場のわき出し吸いこみはないことが電場のGaussの法則と大きく異なる。よって,
 $$\int_S\bm B\cdot \bm n\,dS=\sum_{S\text 内}M=0$$
 と表せる。

 ここで単位を確認しよう。磁荷の単位は\unit{\ruby{Wb}{\text{ウェーバー}}}
 \kyakuchuu{本来なら\unit{C}と並ぶ重要な電磁気学の単位であるが,磁荷が存在しないので\unit{C}ほど頻繁に見ないかもしれない}
 で面積は\unit{m^2}なので\unit{T}${}={}$\unit{Wb/m^2}と表せる。

 \section{電磁場の法則\protect\ajMaru{4}}

 ここでは電流の周りに生じる取り囲み磁場について表現しておく。

  \subsection{Amp\'{e}reの法則}

  定性的には%図
  $I$の向きに右ねじを回して進む向きになるような,$H$のうずまきができる。
  %図

  これを式に表すと以下のようになる。
  $$\oint_C\bm H\cdot dr=\int_S\bm i\cdot\bm n\,dS$$
  ここで$C$は任意のループで左辺は$C$ 1周に沿って単位磁荷を運ぶとき,磁場のする仕事で
  右辺は%$C$で囲まれた
  $C$をふちとする面$S$を貫く全電流である。
  ここで面$S$は$C$をふちとすれば任意で,閉曲面を2つに分けた様なものを想像して,
  例えば封のしていないビニール袋でビニール袋のふちが$C$でビニール袋が面$S$というと分かりやすいだろうか?
  ただし特に大学入試の問題では面を複雑に取ると計算が難しくなるだけなので
  ちょうどシャボン玉のようにループにくっついた面を考えるのがよい。

  しかしここで疑問が生じなければおかしい。なぜなら左辺は$C$ 1周に沿って単位磁荷を運ぶとき,磁場のする仕事と書いたが
  回る方向によって正負が変わってしまう。右辺も単位法線ベクトル$\bm n$は方向を考えると2つ存在するので
  どちらを取るかで正負が変わる。なのでここは数学で頻繁に使われる右手系を使って,$\bm n$の方向に右手の親指を立てて
  他の指の向く方向が$dr$の正方向である。
  もちろん$dr$から先に決めてもいいし,右ねじを$dr$方向に回したときに進む方向を$\bm n$方向にするとしても同じことである。

  \subsection{無限に長い直線電流の周りの磁場}

  %\begin{wrapfigure}{l}{7zw}
  %\input{hi}
   %\end{wrapfigure}
   対称性より直線状どこから見ても,同じ磁場が等方的に取り囲む。よって磁場は円になりループ$C$を半径$r$の円に取ると,
   磁場$H(r)$は$H(r)\cdot 2\pi r=+I\Longleftrightarrow H(r)=\bunsuu{I}{2\pi r}$

  \subsection{無限に長いソレノイドコイルの電流の周りの磁場}

  実際のソレノイドコイルは一本の導線を巻いていくので軸に平行な電流もあるはずだがここでは理想化し,無視する。

  まず当然だが磁場にも重ね合わせの原理は成り立つので円電流が作る磁場を重ねると

  ソレノイド途中放棄


  \section{Biot-Savart(ビオ・サバール)の法則}

  Ampére の法則は積分法則なので任意の点の磁場を求められるのはきわめて対称性ときに限られる。

  よってそうでないときは電流を細かくした電流素片\kyakuchuu{実際には細かくしたら電流は流れなくなるので仮想的なもの}
  が任意の点に作る磁場の法則(Biot-Savart の法則)と重ね合わせの原理を用いる。

  Biot-Savartの法則はもともとは実験的に発見されたが,Ampéreの法則を書き換えたものと考えて欲しい。

  電流素片$I\,ds$が,この素片から任意の位置ベクトル$\bm r$の位置に作る磁場は
  $$dH=\bunsuu{1}{4\pi}\,\bunsuu{I\,ds}{|\bm r|^2}\times \bunsuu{\bm r}{|\bm r|}$$

 \section{電磁誘導}

 電磁誘導には次の2種類ある。
 \begin{enumerate}
  \item 磁場中を運動する導体中の自由電子に働くLorentz力の成分による誘導起電力
  \item 磁場の変化の周りに生じる取り囲み電場(誘導電場)による誘導起電力(=電磁場の法則\ajMaru 3)
 \end{enumerate}
 \ajMaru 1についてPQ中に考えた単位正電荷は$\mathrm{Q\to P}$向きに力$|(\bm v\times\bm B)_{\mathrm{PQ}}|$%
 ($\bm v\times\bm B$のPQ方向成分の絶対値という意味で用いたが,数学的に正しい記号ではない)
 を受ける。よって$\mathrm{Q\to P}$へ動くとき,仕事$l|(\bm v\times\bm B)_{\mathrm{PQ}}|$をされることになり,
 これが$\mathrm{Q\to P}$向きに生じる誘導起電力である。

 これによりPQ方向に電荷分布の偏りができ,$\mathrm{Q\to P}$向きの静電場が生じる。

 定常状態での力のつりあい$0=|(\bm v\times\bm B)_{\mathrm{PQ}}|-|\bm E|$

 よってQに対するPの電位は,$\phi_\text{P}-\phi_\text{Q}=+|\bm E|l=l|(\bm v\times\bm B)_{\mathrm{PQ}}|=V$

 つまり大事なのは\textgt{\bfseries 電位差と起電力は全く違う}ということである。

 (ここでは電磁誘導により)電荷に対し仕事をする状態が生じる→電荷分布が生じる→電位差が生じるというプロセスを踏むのである。
 起電力とは単位電荷に対して仕事をする能力で,電位差は考えている部分に生じた静電場から定義されるものである。

 ここは大事なので具体例をあげて違いを説明しよう。
 %%図

 \begin{description}
  \item[電位差がある回路(ここではコンデンサー)]電荷が移動して電位差を解消しようとし,解消すると電荷は移動しない
  \item[起電力がある回路(ここでは電池で考える)]電荷が移動して電位差を解消しようとするが,%
	     {\bfseries 起電力が電荷を移動させて電場を一定に保つので電位差は変わらない}
 \end{description}

  \subsection{ローレンツ変換}

  ここでまたLorentz力について考えてみよう。$\bm F=q(\bm E+\bm v\times\bm B)$という式をよく見て欲しい。
  速度$\bm v$は見る座標系によって違う。
  極端なことを言えば静止座標系で見たとき速度$\bm v$で動いてる物体は速度$\bm v$で動いている座標系から見ると
  静止しているように見える。

  よって,例えば一様磁場の中に(電場は無いとする)電荷粒子を放り込めば$q\bm v\times\bm B$を向心力とした
  等速円運動が観測されるがこの現象を粒子と同じ速度で動く座標系から見ると粒子は静止して見える。
  また,この座標系は加速度をもっているので慣性力(遠心力)が働くはずである。しかしLorentz力は\textbf 0だから
  力はつりあっていないはずである。

  現在の物理学ではローレンツ変換といい電場と磁場は見る座標系によって何らかの関係性があるとされている。
  なので,前の例の遠心力はローレンツ変換により現れる誘導電場から働く力とつり合う。

  よって電磁誘導\ajMaru 2は図のようにあくまで座標系の違いであり電磁誘導\ajMaru 1と全く同じ現象である。

  \section{電磁場の法則\protect\ajMaru{3}}

  磁場の変化の周りに電場が取り囲む。

  これを式に表すと以下のようになる。
  $$\oint_C\bm E\cdot dr=-\bunsuu{d}{dt}\int_S\bm B\cdot\bm n\,dS$$
  ここで$C$は任意のループで,左辺は$C$\,1周に生じる誘導起電力で,
  右辺は$C$をふちとする面$S$を貫く全磁束である。

  もちろん$dr,\bm{n}$の正方向は電磁場の法則\ajMaru{4}と同じである。

  また普通全磁束を$\Phi$で表し$V=-\bunsuu{d\Phi}{dt}$と書く。これをFaradayの法則という。
  またこのマイナスの符号は次のように解釈できる(レンツの法則)。

  $\Phi$\textgt{の変化を妨げる磁場をつくる電流を流そうとする向き}

  \section{自己・相互誘導}

  Amp\'ereの法則(これを変形したBiot-Savartの法則でもよい)から,電流の作る磁場は電流に比例するので
  $K_1, K_2$を(ふちとする)貫く全磁束は,%
  $%
  \begin{cases}
   \Phi_1=L_{11}I_1+L_{12}I_2\\
   \Phi_2=L_{21}I_1+L_{22}I_2
  \end{cases}$%
  とかける\kyakuchuu{コイルがいくつあっても同じ}。

  $|L_{ij}|$は$K_1, K_2$の形,配置,物質分布により決まる定数。

  このとき $|L_{11}|=L_1$を$K_1$の自己インダクタンスという。

  \hphantom{このとき} $|L_{22}|=L_2$を$K_2$の自己インダクタンスという。

  \hphantom{このとき} $|L_{12}|=|L_{21}|=M$\kyakuchuu{相反定理 証明は難}を$K_1,K_2$の相互インダクタンスという。よって%
  $\begin{cases}
    \Phi_1=+L_{1}I_1+MI_2\\
    \Phi_2=+MI_1+L_{2}I_2
   \end{cases}$%
   とかけることがわかる。ただしインダクタンスの前の符号は面の
   法線ベクトルと磁束の方向により変わりうる。ただし法線ベクトルの方向は任意に決められ,磁束の正方向も任意に決められる\kyakuchuu{嘘??}。
   %記憶が曖昧・・・・・・

   $I_1,I_2$の変化に伴い$K_1,K_2$に生じる誘導起電力は
   $\begin{cases}
     V_1=-\bunsuu{d\Phi_1}{dt}=-L_{1}\bunsuu{dI_1}{dt}-M\bunsuu{dI_2}{dt}\\[4pt]
     V_2=-\bunsuu{d\Phi_2}{dt}=-M\bunsuu{dI_1}{dt}-L_{2}\bunsuu{dI_2}{dt}
    \end{cases}$
    となる。

    またインダクタンスの単位は\unit{Wb/A}${}={}$\unit{\ruby{H}{\text{ヘンリー}}}である。

    またインダクタンスを求めるのは困難なことが多いため高校範囲で求めることが要求されるのはソレノイドコイルのみである。
    %わあああ

  \subsection{(無限に長いと近似できる)ソレノイドコイルのインダクタンス}


  \subsection{自己インダクタンス$L$のコイルの回路における役割}

  $I$と同じ向きを正とした誘導起電力$V$は$V=-L\bunsuu{dI}{dt}$と表せる。たこれは\textgt{起電力}なので
  回路方程式の左辺に書くのが普通である。また,電流$I$の変化を妨げる向きに誘導起電力が生じるので\textgt{\bfseries 逆起電力}
  とも呼ばれる。

  %\subsubsection{コイルに蓄えられる磁気エネルギー}
