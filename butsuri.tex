% catatsuyが若かりし頃に作成しようとして挫折した電磁気の教科書の残骸です
% パブリックドメインですが,もし使うことがあればご一報いただけると嬉しいです
% 連絡先:
% twitter: @catatsuy
% mail: catatsuy@catatsuy.org
\documentclass[a4paper,papersize,openany,12Q]{jsbook}
\usepackage[T1]{fontenc}
\usepackage[utf8]{inputenc}
\usepackage{lmodern}
\usepackage[sc]{mathpazo}
\usepackage[scaled]{helvet}
\usepackage[dvipdfmx]{emathP}
\let\MARU\relax%パッケージ間の衝突回避
\usepackage[samepage]{emathMw}
\usepackage[deluxe,expert]{otf}
\usepackage{textcomp,wrapfig,okumacro,catatsuy}
\begin{document}
\resettagform%emathの数式番号の設定を解除
\frontmatter
\chapter{物理学への導入}
\section{物理の勉強の仕方}

基本原理に基づいて,その原理を理解し(人に教授することのできるレベル),種々の演習を通じて
確認する。自然なものの見方が身につけばもうそれで終わり。

物理において『東大物理』も『医系物理』もなく,『古典物理学』『現代物理学』を合わせて『物理学』
と総称する。また物理において解法などという言葉は使用しない。なぜなら,\textgt{\bfseries 自然なものの見方が身に
ついた人からみれば,その方針が自然に出てくるから}である。

我々が日本語を話す時にいちいち考えて話さないのと同様に,
物理においてもそういった自然に理解できるような見方を身につけることが重要。
\subsection{古典物理学(19世紀までの物理学)の世界観}

古典物理学の世界観は“素朴な”素粒子論\kyakuchuu{現代物理学においては素粒子論という発想が古典物理学とは異なる}
であり,これが言わんとしていることは
「生命だろうが,心(スピリチュアルな世界)だろうが,
すべては究極粒子の振る舞いによって表現される」
ということ。

例えば,我々が考えていることは脳内の素粒子の振る舞い(運動)によって決定されるといったこと。

ただし,今までの内容はあくまでも前提(人間のアイデアの1つ)にすぎず,こういった前提をすることによって
「つじつまが合いそうだ」という発想から生まれたにすぎない。だから,『世界観』という表現を使用する。
学問をする上で,大切な事は「その学問が何を前提としているのか」や「どういった発想を基に,
何を目標としているのか」をとらえることである。当然そこには善し悪しといったレベルの議論は存在
しない(どれも1つの世界観に過ず,絶対的なものではない)。
\subsection{自然科学について}

自然科学は自然界を見るときの基本的な世界観でつじつまが合わなければもちろん修正される
\kyakuchuu{事実,古典力学的な世界観は19世紀ごろに修正された%
(ただし現象をマクロに見るときには依然として古典力学は有用である)}。

絶対的なものではないが,実際問題として上手くいっているので,使っているだけである。
%“素朴な”素粒子論は単純明快
すべての自然現象は素粒子の振る舞いとしてとらえられる。
その真偽判定は実証(現実との照らし合わせ)のみによる。
よって,有限な例についてしか,示すことが出来ない!
(何度もいうようだが,絶対的ではない。)

しかし物理の表現を文学的な言葉で行うと(例えば日本語)分からない人が出てくるし,
本来同じ概念として捉えられるべき概念が人によって異なったものに受け取られる可能性がある
逆に言えば,文学的な言葉はそのあいまいさ故に,多種多様な解釈を可能にし,人類の精神活動を豊かな
ものにするのに重要な役割をしたともいえる。しかしそれだと議論にならないので,
表現は定義ができるだけ正確なものを用いなければならない。

それを解決するために人間が作った言語。それが“数学”である。

これで『なぜ数学は必要か』という質問の答えは明らかになった。
万人が納得するような言語が必要だったからである。

だから自然科学の世界においては,その世界の言葉である“数学”ができないと大きなハンデを
負うことになるのである。

\begin{fullwidthpage}
 \tableofcontents
\end{fullwidthpage}
\mainmatter
\chapter{電磁気学への導入}
\section{場の概念の導入}

素朴な素粒子論に基づく古典物理学における力は次の2種類(量子力学レベルでは他に弱い力・強い力の合計4種類がある)

\begin{itemize}
\item \kintou{4zw}{重力}:質量に働く力%重力質量
\item 電磁気力:電荷に働く力
\end{itemize}

%\begin{wrapfigure}{l}{7zw}
万有引力の式$G\bunsuu{m_1m_2}{r^2}$\marginpar{\begin{zahyou*}[ul=3zw](-1.6,1.1)(-0.2,.5)%
\tenretu*{A(-1,0);B(1,0);C(-.7,0);D(.7,0)}
\Hasen{\A\B}
\ArrowLine\A\C
\ArrowLine\B\D
\Put\A[nw]{$m_1$}
\Put\B[ne]{$m_2$}
\HenKo[0]\B\A{$r$}
\kuromaru{\A;\B}
\end{zahyou*}
}%\end{wrapfigure}
は力学のときに勉強したはずである。
しかしここでもう一度この式を良く考えてみよう。

$m_2$を主体とすると$m_1, r$は相手から影響を受ける情報である。
この情報は例えば$m_1$の位置が変わったり
($m_2$を主体と見ているので,もちろん相対的な変化),
また$m_1$の質量が変わる(これは非常に起こりにくいことだが)とき,
これらの情報は{\bfseries 瞬時}に$m_2$に伝わるのかを考えて欲しい。

重力では実験(観測)は難しいが\kyakuchuu{宇宙空間で星が爆発する等,特殊な場合には発生することが考えられ
ていてる。そのときに発生すると見られる重力波を検出しようと試みる研究者もいる。しかし未だに発見されていない},
電磁気ではこのような実験は容易にできる。
そして結論は瞬時には伝わらず,{\bfseries 時間差}があることが分かっている。

よって以上の議論により万有引力の式の表記はまずいことが分かった。

ではどうすればよいのか。

そこで人間が考えたことが『\textgt{\bfseries 場}』の概念の導入である。

\subsection{場とは}%\label{場とは}
粒子に力を感じさせるような性質が空間にあると考え,
粒子は力の場から力を受けていると考えることが場の概念である。

そして力の場のでき方が粒子と何らかの法則性を持っていると考えると現象が説明できる。

この場の概念は受験生を苦しませることが多い。しかしよく考えてみて欲しい。
我々は空間をほとんどの場合幾何的(座標や住所と考えればよい)な性質だけでは
考えていない。

例えば,明るい・暖かい・にぎやかな・いい感じ・気味の悪いなど空間に何らかの性質を考えている
ことの方が圧倒的に多い。このとき空間にはその性質の場があるという。

そして場には大きく分けて以下の2つがある。

\begin{itemize}
\item スカラー場:明るさの場・温度の場等
\item ベクトル場:風や水の{\bfseries 流れ}の場
\end{itemize}

\subsection{重力・電磁気力の表現}

\subsubsection{重力の表現}

$m$の位置の重力場が$\bm C$の時,重力は$\bm{F} =m\bm{C}$と表せる。

ちなみに地表付近の重力場は近似的に鉛直下向きに一様で大きさ$\varg$である。

\subsubsection{電磁気力の表現}

電場$\bm E (\bm r,t)$,磁束密度$\bm B (\bm r,t)$の位置$\bm r$に電荷$q$がある時,電磁気力は
\[\bm{}F=q(\bm{E} +\bm{v}\times \bm{B})\]
と考えれば電磁気のあらゆる現象が説明でき,
この力をLorentz(ローレンツ)力と呼ぶ。ただしLorentz力というと$q\bm v\times \bm B$のみを
指すことが高校の参考書等では多い。しかし本書では$\bm E$も含めたこの力をLorentz力と呼ぶことにする。

\subsection{電荷の特徴}

電荷というものを考えてみよう。古典物理では素朴な素粒子論を基礎とするので
{\bfseries 電荷は素粒子であると考えると分かりやすい}\kyakuchuu{あくまでも人間のideaである}。
\begin{enumerate}
\item 電荷には最小単位(電気素量$e=1.6\times 10^{-19}\unit C$)が存在し,この整数倍の電荷しか存在しない
\item 孤立系の電荷は保存する(電荷保存則)
\end{enumerate}

\ajMaru 1に関して補足しておく。quarkの電荷の大きさは$\pm\bunsuu{e}{3}, \pm\bunsuu23e$であるが,
quarkは2個や3個の組み合わせの粒子として存在し,quark単体は現実世界には出てこない。よって電荷の合計
は必ず$e$の整数倍になる。それがquarkが発見された今でも電気素量$e$の値を変えていない理由である。

\ajMaru 2にも補足。これは同時刻・同位置に正負両電荷が現れ(対生成),
また逆に同時刻・同位置に正負両電荷が消滅する
(対消滅)という現象が起こることは否定していない\kyakuchuu{事実起こる}。
また宇宙全体は一つの孤立系であり,宇宙全体の電荷は保存されていると考えられている。

\subsection{電磁場の法則}

場とはに書いたようにベクトル場を理解しづらいのならば水の流れを
想像すれば分かりやすい。なので以下の説明は水の流れを思い浮かべて読んで欲しい。

またベクトル場には次の2つがある。
\begin{enumerate}
\item わき出し吸いこみのある場()数学でのdivの概念であるが,今慌ててやる必要は無い
\item 回転の場(取り囲んでいる場)(洗濯機)数学でのrotの概念である
\end{enumerate}
%\begin{tabular}{ccc}
%\begin{zahyou*}[ul=2.5zw](-1,1)(-1,1)%
\kyokuTyoku(1,0)\A%
\kyokuTyoku(1,45)\B%
\kyokuTyoku(1,90)\C%
\kyokuTyoku(1,135)\D%
\kyokuTyoku(1,180)\E%
\kyokuTyoku(1,225)\F%
\kyokuTyoku(1,270)\G%
\kyokuTyoku(1,315)\H%
\ArrowLine\O\A
\ArrowLine\O\B
\ArrowLine\O\C
\ArrowLine\O\D
\ArrowLine\O\E
\ArrowLine\O\F
\ArrowLine\O\G
\ArrowLine\O\H
\end{zahyou*}
\hfil\mbox{} & \input{divv}\hfil\mbox{} & \begin{zahyou*}[ul=2.5zw](-1,1)(-1,1)
\Put\O{\Daen\O{1}{.6}}
\Put\O{\Daenko<yazirusi=a>{1}{.6}{0}{80}}
\Put\O{\Daenko<yazirusi=a>{1}{.6}{-180}{-100}}
\end{zahyou*}
%\begin{zahyou*}[ul=2.5zw](-1,4)(-3,1)
%\tenretu*{A(1,-1)}
%\Put\A{\rotatebox{45}{\Daen\O{2}{1}}}
%\end{zahyou}
\hfil\mbox{} \\ 
%\mbox{}\hfil\textgt{わき出し}\mbox{}\hfil & \mbox{}\hfil\textgt{吸い込み}\hfil\mbox{} & \mbox{}\hfil\textgt{回転}\hfil\mbox{} \\ 
%\end{tabular}
\noindent\hfil\begin{zahyou*}[ul=2.5zw](-1,1)(-1,1)%
\kyokuTyoku(1,0)\A%
\kyokuTyoku(1,45)\B%
\kyokuTyoku(1,90)\C%
\kyokuTyoku(1,135)\D%
\kyokuTyoku(1,180)\E%
\kyokuTyoku(1,225)\F%
\kyokuTyoku(1,270)\G%
\kyokuTyoku(1,315)\H%
\ArrowLine\O\A
\ArrowLine\O\B
\ArrowLine\O\C
\ArrowLine\O\D
\ArrowLine\O\E
\ArrowLine\O\F
\ArrowLine\O\G
\ArrowLine\O\H
\end{zahyou*}
\hfil \input{divv}\hfil \begin{zahyou*}[ul=2.5zw](-1,1)(-1,1)
\Put\O{\Daen\O{1}{.6}}
\Put\O{\Daenko<yazirusi=a>{1}{.6}{0}{80}}
\Put\O{\Daenko<yazirusi=a>{1}{.6}{-180}{-100}}
\end{zahyou*}
%\begin{zahyou*}[ul=2.5zw](-1,4)(-3,1)
%\tenretu*{A(1,-1)}
%\Put\A{\rotatebox{45}{\Daen\O{2}{1}}}
%\end{zahyou}
\hfil \\ 
\hfil\textgt{わき出し}\hfil \textgt{吸い込み} \hfil \textgt{ 回転 }\hfil \\ 

\noindent 次節ではこの2つの場が発生する原因について書く。

\subsubsection{電磁場の法則}

以下のことは法則なのでWhy?と言ってはいけない。また以下のことを定式化したものがMaxwell方程式である。
\begin{enumerate}
\item 電場は\begin{tabular}{@{\,}l@{}}
正電荷からわき出て \\ 
負電荷に吸い込まれる \\ 
\end{tabular}
\item 磁場は\begin{tabular}{@{\,}c@{ }l@{\,}}
N&磁荷からわき出て \\
%\rlap{ S}\phantom{N }
S&磁荷に吸い込まれる \\
\end{tabular}\kyakuchuu{単磁荷(magnetic monopole)は発見されていないので,存在しないと考えればよい}
\item 電場は$\left\{\begin{array}{@{\,}c@{\,}}
磁場の時間変化\\
(磁荷の流れ)\\
\end{array}\right\}$の周りに取り囲む
\item 磁場は$\left\{\begin{array}{@{\,}c@{\,}}
電場の時間変化\\ 
電流 \\ 
\end{array}\right\}$の周りに取り囲む
\end{enumerate}

\subsection{電場のGaussの法則}\label{gauss}

電磁場の法則\ajMaru 1を定式化しよう。

任意の閉曲面\kyakuchuu{世界を外界と内界に分ける曲面。袋を閉じたビニール袋を思い浮かべればよいだろうか?}
$S$を考えて,
\begin{center}
\textgt{\bfseries 任意の閉曲面$\bm S$の表面から出て行く電場の総和}
$\bm \propto$
$\bm S$\textgt{\bfseries 内の全電気量}
\end{center}

である。これを定式化しよう。電場は閉曲面の表面で一様とは{\bfseries 限らない}ので
積分計算が必要となる。

\begin{wrapfigure}{l}{5zw}
\begin{zahyou*}[ul=2.5zw](-1,1.2)(-.1,1.6)
\Put\O{\Daen\O{.25}{.1}}
\tenretu*{n(0,1);E(1,1.4);S(-.25,0)}
\ArrowLine\O\n
\ArrowLine\O\E
\Put\n[nw]{$\bm n$}
\Put\E[ne]{$\bm E$}
\Put\S[w]{$dS$}
\Kakukigou\E\O\n{$\theta$}
\end{zahyou*}

\end{wrapfigure}
よって微小表面$dS$上で考える。微小表面$dS$上では,電場は一様とみなせる。
この面を内→外へ貫く\kyakuchuu{もちろんこの向きを正とするという意味である}
電場の量を次のように表す。

雨が鉛直下方に降っているとき,傘を傾けると防げる雨の量は
鉛直上方と傘の先端(数学的には傘の面の法線ベクトル)の成す角を$\theta$とすると$\cos\theta$に比例して減る。
よって雨が鉛直下方から傾いている場合は傘を雨の降る方向に対して先端を平行(つまり傘の面は垂直)にすると最も防げる。

よって$dS$の内→外へ向かう{\bfseries 単位法線ベクトル}$\bm n$を
考え$\bm n$と$\bm E$の成す角を$\theta$とすると電場の量は

$E\,dS\cos\theta =\bm E\cdot \bm n\,dS$

これをすべて足せば任意の閉曲面$S$の表面から出て行く電場の総和が分かるので

$$\int_S\bm E\cdot \bm n\,dS\propto\sum_{S\text 内}Q$$

と表せる。

ここで単位を確認しよう。電場$\bm E$は単位電荷あたりに働く力だから電場$\bm E$の単位は\unit{N/C}である。
また$dS$は\unit{m^2}なので左辺は\unit{N\cdot m^2/C}である。右辺はもちろん\unit{C}だから
比例定数を右辺につけるとしたら単位は\unit{N\cdot m^2/C^2}である。

しかし我々が用いる単位系では比例定数は歴史的経緯から$\bunsuu{1}{\epsilon_0}$と表すので

$$\int_S\bm E\cdot \bm n\,dS=\bunsuu{1}{\epsilon_0}\sum_{S\text 内}Q$$

と表す\kyakuchuu{高校範囲外だが電束密度$\bm D$を使うと比例定数は必要ない}。

この法則を利用して高校範囲で使用する定理を導こう。

\subsubsection{静止した点電荷の周りの電場}\label{G_ex1}

空間は一様・等方でありどの場所も全く対等である\kyakuchuu{ただし
物質分布があると対等ではなくなる}。よって電場は点電荷の周りへ{\bfseries 等方的}にわき出る。

\begin{wrapfigure}{l}{10zw}
\begin{zahyou*}[ul=3.5zw](-1.4,1)(-1,1)%
\kyokuTyoku(1,0)\A%
\kyokuTyoku(1,45)\B%
\kyokuTyoku(1,90)\C%
\kyokuTyoku(1,135)\D%
\kyokuTyoku(1,180)\E%
\kyokuTyoku(1,225)\F%
\kyokuTyoku(1,270)\G%
\kyokuTyoku(1,315)\H%
\ArrowLine\O\A
\ArrowLine\O\B
\ArrowLine\O\C
\ArrowLine\O\D
\ArrowLine\O\E
\ArrowLine\O\F
\ArrowLine\O\G
\ArrowLine\O\H
\En{\O}{.8}
\Put\O[sw]{\scalebox{.7}{Q}}
\kuromaru{\O}
\tenretu*{Z(.8,0)}
\HenKo[0]<Agezoko=-3pt>\Z\O{$r$}
\Put\B[se]{$\bm{E}(r)$}
\end{zahyou*}

\end{wrapfigure}
よって点電荷$Q$から$r$の位置の電場を$E(r)$と表すと半径$r$の球面$S$にGaussの法則を用いて,

$E(r)\times 4\pi r^2=\bunsuu{Q}{\epsilon _0}\Leftrightarrow \bm{E(r)=\bunsuu{1}{4\pi\epsilon_0}\,\bunsuu{Q}{r^2}}$

太字の式は電場のCoulombの法則と呼ばれ公式である。

またこれをベクトルで表すと
$$\bm E(r)=\bunsuu{1}{4\pi\epsilon_0}\,\bunsuu{Q}{r^2}\,\bunsuu{\bm r}{r}
\hfil\left (k_0=\bunsuu{1}{4\pi\epsilon_0}=9.0\times 10^9\unit{N\cdot m^2/C^2}\right )$$
$k_0$はCoulomb力の比例定数と呼ばれている。

ただし静止しているとき自分自身の電荷が作り出した電場の影響は受けないことが実験的に証明されている。

\begin{wrapfigure}{l}{22zw}
\begin{zahyou*}[ul=3.5zw](-3,1.1)(0,.4)%
\tenretu*{A(-1.2,0);B(1.2,0);C(-.8,0);D(.8,0)}
\Hasen{\A\B}
\ArrowLine\C\A
\ArrowLine\D\B
\Put\C[n]{\scalebox{.7}{$Q_1$}}
\Put\D[n]{\scalebox{.7}{$Q_2$}}
\Put\A[nw]{\scalebox{.7}{$E_2=k_0\bunsuu{Q_2}{r^2}$}}
\Put\B[ne]{\scalebox{.7}{$E_1=k_0\bunsuu{Q_1}{r^2}$}}
\Put\A[sw]{\scalebox{.7}{$F_1=Q_1E_2=k_0\bunsuu{Q_1Q_2}{r^2}$}}
\Put\B[se]{\scalebox{.7}{$F_2=Q_2E_1=k_0\bunsuu{Q_1Q_2}{r^2}$}}
\HenKo[0]\B\A{$r$}
\kuromaru{\C;\D}
\end{zahyou*}

\end{wrapfigure}
よって点電荷同士に働きあう静電気力(Coulomb力)は左図のようになる($Q_1,Q_2>0$)。

\subsubsection{球対称電荷分布による電場}
\begin{wrapfigure}{l}{10zw}
\begin{zahyou*}[ul=3.5zw](-1.4,1)(-1,1)%
\kyokuTyoku(1.1,0)\A%
\kyokuTyoku(1.1,45)\B%
\kyokuTyoku(1.1,90)\C%
\kyokuTyoku(1.1,135)\D%
\kyokuTyoku(1.1,180)\E%
\kyokuTyoku(1.1,225)\F%
\kyokuTyoku(1.1,270)\G%
\kyokuTyoku(1.1,315)\H%
\kyokuTyoku(.8,0)\a%
\kyokuTyoku(.8,45)\b%
\kyokuTyoku(.8,90)\c%
\kyokuTyoku(.8,135)\d%
\kyokuTyoku(.8,180)\e%
\kyokuTyoku(.8,225)\f%
\kyokuTyoku(.8,270)\g%
\kyokuTyoku(.8,315)\h
\ArrowLine\a\A
\ArrowLine\b\B
\ArrowLine\c\C
\ArrowLine\d\D
\ArrowLine\e\E
\ArrowLine\f\F
\ArrowLine\g\G
\ArrowLine\h\H
\En{\O}{.8}
\En{\O}{.7}
\En{\O}{.55}
\Put\O[sw]{O}
\kuromaru{\O}
\HenKo<henkoH=5pt>\a\O{$\,r\,$}
\Put\B[se]{$\bm{E}(r)$}
\end{zahyou*}

\end{wrapfigure}
対称性より点Oから見ると,等方的に電場が生じる。
図の球面$S$にGaussの法則を用いると,
$$E(r)=\bunsuu{1}{4\pi\epsilon_0}\,\bunsuu{\tsum{S\text{内}}{}Q}{r^2}$$
となる。よって点Oに電荷$\tsum{S\text{内}}{}Q$が点電荷として存在するときの
電場と同じである\kyakuchuu{万有引力の式の距離が物体の中心の距離であるのはこのような理由による}。

\subsubsection{無限に長い直線状に一様分布した電荷の周りの電場}

対称性から直線状どこから見ても同じ電場が直線に垂直\kyakuchuu{垂直でないと,見るところによって電場が変わる}に
等方的に生じる。よって図の円柱面SにGaussの法則を用いて
$$E(r)\times 2\pi rh=\bunsuu{1}{\epsilon_0}\rho h\Leftrightarrow \bm{E(r)=\bunsuu{1}{2\pi\epsilon_0}\,\bunsuu{\rho}{r}}$$

\subsubsection{無限に広い平面状に一様分布した電荷の周りの電場}\label{G_ex4}

\begin{wrapfigure}{l}{15zw}
\begin{zahyou*}[ul=2zw](-3.5,4)(-2,2)%
\def\A{(-2.5,.85)}
\def\B{(-3.5,-.85)}
\tenretu*{N(-1,0);M(-1,1.5);L(1,0);P(1,1.5);Q(0,1.5);m(-1,-1.5);p(1,-1.5);S(-1.9,0);T(-1.9,1.8);U(-1.65,0);V(-2.15,0);R(-1.9,-1.8)}
\Subvec\O\A\C
\Subvec\O\B\D
\Put{(0,0)}{\Daenko<hasen=[0.5][0.5]>{1}{.5}{0}{180}}
\Put{(0,0)}{\Daenko{1}{.5}{-180}{0}}
\Put{(0,-1.5)}{\Daenko<hasen=[0.5][0.5]>{1}{.5}{-180}{0}}
\Daen{(0,1.5)}{1}{.5}
\Drawline{\A\B\C\D\A}
\Drawline{\N\M}
\Drawline{\L\P}
\Drawline{\U\V}
\Drawlines<sensyu=\dashline[250]{.06}>{\L\p;\N\m}
\kuromaru{\Q}
\HenKo<henkoH=5pt>\L\P{$a$}
\ArrowLine\S\T
\Tyokkakukigou(3.5)\U\S\T
\Put\T[ne]{$E$}
\ArrowLine<sensyu=\dashline[250]{.06}>\S\R
\end{zahyou*}

\end{wrapfigure}

対称性より平面状どこから見ても同じ電場が面に垂直に生じる。図の円柱面$S$にGaussの法則を用いて,
底面積$A$として$$E\times 2A=\bunsuu{1}{\epsilon_0}\sigma A\Leftrightarrow E=\bunsuu{\sigma}{2\epsilon_0}$$

これは公式として使ってよく,特にコンデンサー%参照
のところで非常に重要な公式である。

\ref{G_ex1}\textasciitilde\ref{G_ex4}から分かるように,Gaussの法則は積分計算をしなければならないので
任意の点の電場を求められるのはきわめて対称性の強い電荷分布のときに限られる。



\subsection{電場の重ね合わせ}

前節の議論でGaussの法則から電場が求まるのは,きわめて対称性の強い電荷分布のときに限られることがわかった。
しかしこれでは例えば2つの静止した点電荷があったときの電場は求められないことになる。
よってこの電場を求めるには他の方法を取らざるを得ない。
そこで重ね合わせの原理を利用することになる

重ね合わせの原理とは例えば点電荷$Q_1, Q_2$以外の電荷は
無視できる世界での\kyakuchuu{“こんなことありえない”と思う人も
いるかもしれないが,「$Q_1, Q_2$以外の電荷は無限遠方とみなせる」という意味である}
点Pを考える。

$Q_1$のみが存在するときのPの電場を$\bm{E}_1$,$Q_2$のときを$\bm{E}_2$とすると,$Q_1, Q_2$が存在するときの
点Pでの電場$\bm E$は$\bm E=\bm{E}_1+\bm{E}_2$のベクトル和になる。

電荷が何個あっても電場はベクトル和であることが実験的に示されている。

\subsubsection{重ね合わせの原理からGaussの法則\ref{G_ex4}を求める}

\begin{wrapfigure}{l}{20zw}
\begin{zahyou*}[ul=2.5zw](-4,5)(-2,4)%
\def\A{(-3,1.5)}%
\def\B{(-4,-1.5)}
\def\Q{(0,4.1)}
\def\N{(2.8,1.3)}
\def\S{(1.5,-.715)}
\tenretu{P(0,3)e}
\Subvec\O\A\C
\Subvec\O\B\D
\Subvec\O\S\T
\Subvec\P\S\U
\Mulvec{1.2}\U\R
\Addvec\R\S\L
\ArrowLine\P\L
\Subvec\P\T\V
\Mulvec{1.2}\V\J
\Addvec\J\T\K
\ArrowLine\P\K
\Daen*[0.3]{\O}{2.1}{1.1}
\Daen*[0.0]{\O}{2}{1}
\kuromaru{\O;\P;\S;\T}
\Dashline{0.1}{\O\P\Q}
\Dashline{0.1}{\S\P\T}
\Drawline{\A\B\C\D\A}%
\Drawline{\S\T}
\HenKo[0]\O\S{$r$}
\HenKo[0]\S\P{$\sqrt{r^2+a^2}$}
\HenKo\P\O{$a$}
\Put\O[sw]{O}
\Put\N[se]{$\sigma$}
\touhenkigou[\scalebox{.6}{|}]<1>{\P\L;\P\K}
\end{zahyou*}
\end{wrapfigure}
%{\begin{zahyou*}[ul=.25zw](-2,2)(-2,2)%
%\En*[0.3]{\O}{2}
%\En*[0.0]{\O}{1.6}
%\end{zahyou*}}
灰色部分の電荷は電荷面密度$\sigma$と厚さ$dr$を用いて$\sigma \times 2\pi r\,dr$と表せ,この部分がPに作る電場は$\Vec{OP}$向き\kyakuchuu{ベクトル和を良く考えてみよう}だから,成分を考えて
$$\bunsuu{1}{4\pi\epsilon _0}\,\bunsuu{\sigma\times 2\pi r\,dr}{(\sqrt{r^2+a^2})^2}\times\bunsuu{a}{\sqrt{r^2+a^2}}=\bunsuu{\sigma a}{2\epsilon_0}\,\bunsuu{r\,dr}{(r^2+a^2)^{\frac32}}$$

の$r=0\text\textasciitilde\infty$の和が求める電場であるから,
\begin{align*}
E&=\bunsuu{\sigma a}{2\epsilon_0}\dint{0}{\infty}\bunsuu{r\,dr}{(r^2+a^2)^{\frac32}}\\
{}&=\bunsuu{\sigma a}{2\epsilon_0}\teisekibun{-\bunsuu{1}{\sqrt{r^2+a^2}}}{0}{\infty}=\bunsuu{\sigma}{2\epsilon_0}
\end{align*}

これは当然ながら\ref{G_ex4}のときと同じ結果である。

\chapter{Potential Energyと電位}
 \section{保存力}

 \kenten{一般}に仕事$W=\dint{C}{}\bm F\cdot dr$は,始点Aと終点Bの位置を決めてもそれだけでは決まらず,
 途中経路に依存する\kyakuchuu{摩擦力など例はいくらでもある}。{\bfseries 力の種類によっては},$W$が任意の
 始点,終点に対し,途中経路によらず,始点,終点の{\bfseries \kenten{位置}のみで決まる}。
 このような力を『\textgt{\bfseries 保存力}』という。
 %図
   \subsubsection{地表付近の重力は保存力}

   例えば図のように質量$m$の物体を任意の経路$C$に沿って,地表付近$y_\text{A}\text{\textasciitilde} y_\text{B}$動かす。%
   $\bm F=(0,-m\varg,0), d\bm r=(dx,dy,dz)$とおけるから,仕事$W$は
   $$\dint{C}{}\bm F\cdot dr=\dint{C}{}(-m\varg)\,dy\\
   =-m\varg\dint{C}{}dy=-m\varg(y_\text{B}-y_\text{A})$$
   これは経路$C$によらず,$y_\text{A}, y_\text{B}$の位置のみで決まる。よって地表付近の重力は
   保存力である\kyakuchuu{後で詳説するが地表付近の重力が保存力だから,重力の位置エネルギーが定義できる}。

   \subsubsection{静電気力も保存力}

   重ね合わせの原理より1つの静止点電荷による静電気力が保存力であることを示せばよい。
   %図

   任意の経路$C$はギザギザを極めて小さくとれば,図のような経路の組み合わせにできる。

   A, Bを決めると,$r_1=r_\text{A}, r_N=r_\text{B}$と固定できるが,経路により$r_2\text\textasciitilde r_{N-1}$はいろいろ変わる。
   \begin{align*}
    \dint{C}{}\bm F\cdot dr&=\dint{r_1}{r_2}k_0\bunsuu{Qq}{r^2}\,dr+\dint{r_1}{r_2}k_0\bunsuu{Qq}{r^2}\,dr+\dots +\dint{r_{N-1}}{r_N}k_0\bunsuu{Qq}{r^2}\,dr\\
    {}&=\dint{r_1=r_\text{A}}{r_N=r_\text{B}}k_0\bunsuu{Qq}{r^2}\,dr=-k_0Qq\left(\bunsuu{1}{r_\text{B}}-\bunsuu{1}{r_\text{B}}\right)
   \end{align*}

   これより仕事は$r_\text{A}, r_\text{B}$のみで決まり,途中経路によらないことが示された。

   また保存力はどんな動き方をしてもスタート地点に戻ってきたら
   その仕事の合計は{\bfseries 絶対}に$\bm 0$ということは
   知っておきたい\kyakuchuu{1周積分$\displaystyle\oint\bm F\cdot d\bm r=0$}。

   \subsubsection{Potential Energy}\label{PE}

   力$\bm F$が保存力であるとき,Potential Energyは次のように定義する。

   Aを基準とするBのPotential Energy $U_\text{B}$は%
   $U_\text{B}=-\dint{\text A}{\text B}\bm F\cdot d\bm r$\kyakuchuu{保存力なので始点と終点だけ指定すればよい}

   つまり『\textgt{AからBへ運ぶとき,保存力につりあわせる外力のする仕事}』\hfill \ajKakkoAlph{1} \\
   $\Longleftrightarrow$『\textgt{BからAへ動くとき,保存力自身がする仕事}』\hfill \ajKakkoAlph{2}

   \ajKakkoAlph{1},\ajKakkoAlph{2}どちらでも言えるようにしておくべきである。

   以上より点電荷$Q$による静電気力のPotential Energyは$r_\text{A}$を基準とすると

   $$U=-\dint{r_\text{A}}{r_\text{B}}\left(k_0\bunsuu{Qq}{x^2}\,dx\right)=k_0Qq\left(\bunsuu{1}{r_\text{B}}-\bunsuu{1}{r_\text{A}}\right)$$
   と表せる。

   \subsection{電位}

   電位とは単位電荷あたりの静電気力によるPotential Energyで単位は\unit{J/C}${}={}$\unit{\ruby{V}{\text{ボルト}}}

   前節の議論より$r_\text{A}$に対する$r_\text{B}$の電位$\phi$は,

   $$\phi =\bunsuu{U}{q}=\bm{k_0Q\left(\bunsuu{1}{r_\text{B}}-\bunsuu{1}{r_\text{A}}\right)}$$

   特に$r_\text{A}\to\infty$にとれば$\bm{\phi_\text{B}=k_0\bunsuu{Q}{r_\text{B}}}$

   一般に静電場を$\bm E$としたとき,Aに対するBの電位は$\phi_\text{B}=-\dint{\text A}{\text B}\bm E\cdot d\bm r$

   \subsubsection{一様電場による電位}

   定数での積分計算をしても良いが,\ref{PE}の\ajKakkoAlph{1},\ajKakkoAlph{2}を利用すると,電場を水の流れと見たときに
   “上流側”の方が高電位であることはすぐにわかり,
   それが距離と比例関係にあることもすぐにわかる\kyakuchuu{というよりも分かるように訓練するべきである}。

   \subsubsection{電位の重ね合わせ}

   静電場$\bm E=\bm{E_1}+\bm{E_2}+\cdots +\bm{E_N}$のとき
   $$\phi _\text{B}=-\dint{\mathrm A}{\mathrm B}\bm E\cdot d\bm r=%
   -\dint{\mathrm A}{\mathrm B}\bm{E_1}\cdot d\bm r%
   -\dint{\mathrm A}{\mathrm B}\bm{E_2}\cdot d\bm r%
   -\cdots -\dint{\mathrm A}{\mathrm B}\bm{E_N}\cdot d\bm r
   =\phi_{1B}+\phi_{2B}+\cdots+\phi_{NB}$$
   と表せる。一見すると当たり前のことだが,ベクトル和である電場の重ね合わせに比べ,
   電位はスカラー和であるので計算は電位の方が楽である。

   \subsection{Potential Energyから保存力を求める}

   前節よりPotential Energyはスカラーであるから計算が楽であることが分かった。

   そこで計算が楽なPotential Energyで計算して必要に応じて保存力を逆算することができれば便利である。
   そこでその逆算の方法について考えてみよう。

   Potential Energy $U(x, y,z)$,保存力$\bm F(x, y, z)$とする

   点$\mathrm{A}(x_1,y_1,z_1), \mathrm{B}(x_1+\varDelta x,y_1,z_1)$を考えると

   $$U(x_1+\varDelta x,y_1,z_1)-U(x_1,y_1,z_1)=-\dint{\mathrm A}{\mathrm B}\bm F\cdot d\bm r=-\dint{x_1}{x_1+\varDelta x}F_x\,dx$$

   ただし$F_x$は$\bm F$の$x$成分である。

   ここで$\varDelta x\kinzi 0$のとき

   $$U(x_1+\varDelta x,y_1,z_1)-U(x_1,y_1,z_1)\kinzi -F_x(x_1,y_1,z_1)\times\varDelta x$$

   $\varDelta x\longrightarrow 0$を考えて
   \begin{align*}
    F_x(x_1,y_1,z_1)&=\lim_{\varDelta x\to 0}\left\{-\bunsuu{U(x_1+\varDelta x,y_1,z_1)-U(x_1,y_1,z_1)}{\varDelta x}\right\}\\
    {}&=-\bunsuu{\partial U}{\partial x}\,(x_1,y_1,z_1)
   \end{align*}
   これは偏微分と呼ばれる演算で$x$以外の変数を固定して%
   $x$で微分せよという意味で,$\partial$は$d$とほぼ同じ意味だが$\partial$を使うと偏微分であることを強調できる


   よって\textgt{\bfseries 保存力の$\bm x$成分は,Potential Energyの$\bm x$方向にそった変化の割合の逆符号である}。
   そしてこれは$y, z$方向についても成り立つから,

   $$\bm{F=\left(-\bunsuu{\partial U}{\partial x}, -\bunsuu{\partial U}{\partial y},-\bunsuu{\partial U}{\partial z}\right)=-\mathrm{grad}\,U=-\nabla U}$$
   ただし,これらの記号は現時点で覚える必要はない

   \subsection{Potential Energyを利用して運動の様子を目で見る}

   保存力のみを受ける物体のエネルギー保存則より,
   運動エネルギー(Kinetic Energy)を$K$,Potential Energyを$U$,物体のもつエネルギーを$E$として

   $K+U=E$

   よって\textgt{\bfseries $\bm U$のグラフが分かれば$\bm K$が分かり運動の様子が非常にわかりやすくなる}。

   \section{静電エネルギー}

   静電エネルギーとは考えている電荷分布を作るのに必要な静電気力につりあわせる外力のする仕事。

  \subsection{大きさムシの荷電粒子系の静電エネルギー}

  $Q_1\text\textasciitilde Q_N$が存在するような電荷分布を作る。静電気力は保存力であることから
  {\bfseries 計算しやすい動かし方で計算すればよい}。

  よってまず何もない\kyakuchuu{$Q_1\text\textasciitilde Q_N$は無限遠方}状態から$Q_1$を所定の位置に運ぶとき,
  外力のする仕事$W_1=0$である。

  次に$Q_2$を所定の位置に運ぶと,外力のする仕事は$W_2=Q_2\times k_0\bunsuu{Q_1}{r_{12}}$

  よって$N=2$のときの静電エネルギーは$U_2=W_1+W_2=k_0\bunsuu{Q_1Q_2}{r_{12}}$

  同様に考えていけば$N$個の粒子系の静電エネルギーは,$U_N=\tsum{i<j}{}k_0\bunsuu{Q_iQ_j}{r_{ij}}$

  静電エネルギーはコンデンサー等でも大事な考え方である。

  \subsection{電気力線について}



  \section{導体について}

  \kenten{理想的}な導体は内部に自由電子(伝導電子)
  が無数に存在する物体\kyakuchuu{現実には$\sim 10^{28}\unit{個/m^3}$程度であり,また導体は主に金属のことである}。
  絶縁体(誘電体・不導体)はその逆で内部に自由に動ける電荷がない物体である。

  \subsection{静電場の中に導体を入れる}

  静電場の中に導体を入れると
  静電誘導\kyakuchuu{内部の自由電子が電場から力を受けて移動すること}により現れた電荷の作る電場と
  外部電場の重ね合わせにより,電荷移動が終わったとき,
  \textgt{\bfseries 導体内部の電場は$\bm 0$}%
  \kyakuchuu{というより$\bm 0$でないと電荷が力を受けて動いてしまうので,$\bm 0$になるまで電荷は動き続ける}。
  よって導体内部は等電位で,マクロに見た電荷分布は導体の表面のみである\kyakuchuu{ミクロに見れば内部にももちろんある}。

\chapter{回路}
 \section{コンデンサー}

 導体間に電荷を移動させ保持したとき,これらの導体は{\bfseries コンデンサーを形成する}という。
 このとき移動させた電荷の大きさ$Q$\kyakuchuu{導体間の帯びている電荷の差ではない}をコンデンサーに蓄えられた電荷という。
 また$C=\bunsuu{Q}{V}$を(電気)容量と\textgt{\bfseries 定義}する。また容量の単位は\unit{\ruby{F}{\text{ファラド}}}である。

 よってコンデンサーは回路だけではなく例えば,下敷きで髪をこすっても電荷移動は起こるからコンデンサーであるし,
 雷雲と地面も電荷移動が起こっているからコンデンサーである。

  \subsection{電気容量は定数か?}

  Coulombの法則と重ね合わせの原理より\textgt{\bfseries 金属表面の電荷分布を変えず}に$Q$を2倍にすると,
  電場・電位ともに2倍となる。よって電気容量は極板の形・配置・極板間の物質分布による
  定数とみなせる\textgt{\bfseries 可能性がある}。

  しかし雷雲と地面のように電荷がたまりすぎると放電が起きることがあるので
  コンデンサーで蓄えられる電荷には限界がある。よって実際にはある程度の範囲で定数であることが
  知られており,その証明は大学でやる\kyakuchuu{電気容量が定数であることの証明はまだやっていないことを念頭に入れて欲しい}。

  \subsection{コンデンサーの静電エネルギー}

  静電エネルギーとは『考えている電荷分布を作るのに
  必要な静電気力につりあわせる外力のする仕事』であり,これはもちろんコンデンサーのときにも当てはまる。

  よって電荷に偏りのない状態から電荷を移動させてコンデンサーと同じ電荷分布を作るわけだが,
  一気に電荷$Q$を動かすと電場が複雑に変化し計算できない。
  よって静電気力が保存力であることを利用して計算する方法を考えよう。

  ここで物理ではおなじみの考え方。電位$v$の変化が無視できるくらいの
  微小な電荷$dq$を移動させるのに必要な外力を$q=0\text 〜Q$まで足せば静電エネルギーは出るはずである。

  よって求める外力の大きさは電位の定義から$v\,dq=\bunsuu{q}{C}\,dq$だから$Q=CV$とすると

  $$\bm{U=\dint{0}{Q}\bunsuu{q}{C}\,dq=\bunsuu{Q^2}{2C}=\bunsuu12QV=\bunsuu12CV^2}$$

  となる。これは既知として使ってよい。

  \section{(無限に広いとみなせる)平行金属板コンデンサー}




  \subsection{極板間への物体挿入(導体)}




  \subsection{極板間への物体挿入(誘電体)}




 \section{電流}
 %\subsection{電流の定義}
 %
 大きさは断面を単位時間当たり\kyakuchuu{無論,単位時間中に通る電流が一定である保証は無い}に通る電気量。
 よって単位は\unit{\ruby{A}{\text{アンペア}}}${}={}$\unit{C/s}である。

 そして正方向を正電荷の流れと同じ向き
 \kyakuchuu{つまり負電荷である電子の流れとは{\bfseries 逆向き}}とする。
 これは導線内の電場の向きと同じである。

 ただし電場が生じているなら自由電子はどんどん加速されて速さが大きくなり電流は大きくなるはずである。
 しかし実験的に電流は定常状態になることが分かっているので,熱振動している陽イオンなどに自由電子がぶつかり減速もされる
 と考えられる\kyakuchuu{ただし,\ref{抵抗?}も読んで欲しい}。

 この二つの原因から電流は平衡状態となり,平均移動速度$\overline{v}$は一定になる。

 %図

 $I=enS\overline{v}$であることが容易に分かる。

 \section{電気抵抗の定義と特徴}

 電気抵抗を$R=\bunsuu{V}{I}$と定義する(単位は\unit{\ruby{$\Omega$}{\text{オーム}}}${}={}$\unit{V/A})。

 この電気抵抗は電場$E$の大きさが2倍になれば,電流$I$も自由電子の加速度が2倍になるのだから2倍になると考えられ,
 また電圧(電位と同じだが特に電気回路では電圧と呼ぶことが多い)$V$も2倍になると考えられる。

 よって電気抵抗$R$は一定であると考えられ,またそれは実験的にも正しいことが分かっている。
 また$R$は
 \begin{enumerate}
  \item 物質の種類(主に$n$の違い)
  \item 金属の温度(陽イオンの熱振動が自由電子の衝突に影響していると考えられる)
  \item 長さ$l$,断面積$S$($S$が違うと$I$は変化し,$I$が同じ時は$E$が同じなので,$l$が違うと$V$は異なる)
 \end{enumerate}
 に影響されると考えられる。

 \ajMaru{1},\ajMaru{2}を固定したときの,$R$の\ajMaru{3}の依存性

 $S$を2倍にする。

 %図

 $l$を2倍にする。

 %図

 よって$R\propto\bunsuu{l}{S}$と考えられ,$R=\rho\bunsuu{l}{S}$をかくとき,$\rho\unit{\Omega\cdot m}$を抵抗率という。
 そして抵抗率$\rho$が\ajMaru{1},\ajMaru{2}に依存すると考えられる。

 \begin{description}
  \item[$\bm\rho$の\ajMaru{1}による依存性]$n$の大きい金属の方が$\rho$は小さい傾向にある。
  \item[$\bm\rho$の\ajMaru{2}による依存性]金属では,温度により自由に動けない電子が自由に動けるようになったりは
	     ほとんどしない\kyakuchuu{半導体では温度の上昇と共に$n$が大きくなり,この影響により$\rho$は小さくなる傾向がある}。
	     よって$n$は温度にはほとんど影響を受けない。しかし温度があがると\kenten{\bfseries 陽イオンなどの
	     熱振動が激しくなり},自由電子の散乱が起こりやすくなり,$\rho$は大きくなる傾向にある。
 \end{description}

  \subsection{しかし,圏点部分をよく考えてみよう!}\label{抵抗?}

  実はこれは大嘘である。

  randomな振動をしている陽イオンは一定間隔で分布していて,randomな振動をしているのだから,
  陽イオン同士には近づいたり,遠ざかったりするものがrandomにある。よって電子から見た平均の格子間隔は変わらないはず
  (おかしいと気づかないならもう一度見直してよく考えよう)。そうするとぶつかりやすさも変わらないはずである。古典物理学的に
  言うなら,自由電子の振動が激しくなってすぐに衝突してしまうという方が適切であるが,陽イオンの熱振動で議論するのが普通である。

  さらにいうと実験的には絶対零度に近い温度では抵抗が0になる(超伝導現象)が起こることが分かっている。
  よって単純に電子が陽イオンに『\textgt{\bfseries ぶつかる}』というイメージは\textgt{\bfseries 正確でない}。

  実際,この抵抗の問題は量子力学がない時代には超難問とされてきた。
  そして現在これは量子力学でないと説明不可能であることが知られていて,
  その量子力学によると抵抗の原因は結晶格子の乱雑さ(規則正しければ0)によるものとされている。

  ちなみに半導体は温度が上がると動ける自由電子が増加するので$n$が増加する。よって温度が上がると抵抗率は減少する。

  \subsection{Joule熱について}

  単位時間あたりに導線内の自由電子は,電場から$IV$の仕事をされるが,これを陽イオンなどとの衝突により
  それらの熱振動を激しくするのに使われる\kyakuchuu{$IV$の仕事をもらっているのに平均移動速度は不変なのはこういう理由からである}。

  よって単位時間あたりに失われるenergyは$P=IV=RI^2=V^2/R\unit{J/s}$

  \section{(準定常電流)回路の考え方}

  準定常電流回路とは電磁場の発生が無視でき,それにより回路から空間に持ち出されるenergyが無視できる回路のこと。
  そのように仮定すれば電荷電流の分布を楽に求めることができる。

  \subsection{回路を解く方法}
  \begin{itemize}
   \item 電荷・電流を仮定(もちろん物理法則に反しないように)
   \item 以下の回路方程式で求める
	 \begin{itemize}
	  \item 孤立系の電荷保存
	  \item 任意のループ1周に対し,(ループ1周あたりの起電力)${}={}$(ループ1周あたりの電圧降下)
	 \end{itemize}
  \end{itemize}

  ただし起電力とは電荷に対して仕事を\textgt{\bfseries 恒常的}\kyakuchuu{ここが電位差と違う}にする能力のことである。
  また電圧とは電位差のことで特に回路で大変よく使われる。
  その理由は『ぐるぐる』電場上のある点Aを考えると『ぐるぐる』電場を電荷が点Aから1周して点Aに戻ってきたときに
  電荷は仕事をもらっている。よってこのとき電荷に働く力は保存力ではない。よって電位が定義できないので電位差が使われる。%??
  %過度現象とか例題を載せたほうが良いのか?

  \subsection{energy収支}

  運動方程式の両辺にスカラー積で速度$v$をかけると仕事率の式に変わる。実は回路方程式にも似たようなことができる。

  両辺に電流$i$をかけると\kyakuchuu{もちろん回路方程式はスカラー}仕事率の式にかわる。
  それを時刻$t=0\text 〜\infty$において積分すれば仕事になることは明らかである。

  \section{電磁場中での荷電粒子の運動}
  荷電粒子$q$に電場$\bm E$,磁束密度$\bm B$の元で,実験的に以下のような力$\bm F$が及ぼされることが分かっている。
  $$\bm F=q(\bm E+\bm v\times\bm B)$$
  この力はLorentz力と呼ばれる。またベクトル積の特徴から$\bm v\times\bm B$の項は$\bm v$と直行しているので,
  電荷に対して{\bfseries 仕事をしない}。また$\bm B$方向には磁場から電荷が受ける力は働かない。

  よって荷電粒子の運動を求めるにはLorentz力と運動方程式から解けばよいが電場や磁場が時間的に変化すると
  複雑な積分計算となるから,特に高校範囲では一様電場・磁場になる。

  \subsection{電磁場中での電荷と磁荷の運動}

  ここで電荷,磁荷が電磁場から受ける力をまとめると
  \begin{gather*}
   \bm F=q(\bm E+\bm v\times\bm B)\\
   \bm F=m(\bm H-\bm v\times\bm D)
  \end{gather*}
  と表せる。ただし$M$は質量ではなくmagnetic(磁荷)のことである。

  ここで$H,D$はそれぞれ“磁場”,“電束密度”と呼ばれ,それぞれ電場,磁束密度に対応する値である。

  電場・電束密度,磁場・磁束密度は単なる測り方の違いでそれぞれ真空中では$\epsilon_0 E=D, \mu_0 H=B$の関係式が成立している。

  また$\epsilon_0, \mu_0$は真空の誘電率,真空の透磁率と呼ばれる定数である。

  この式を見ればなぜ電気と磁気が同じ“電磁”気学となったのかが少しわかるだろう%
  \kyakuchuu{そうなるとなぜ磁荷が無いのか大変疑問である}。

 \section{電流が磁場から受ける力}

 歴史的には電流が磁場から受ける力が最初に観測され,それについて研究をしてLorentz力は発見されたが,
 ここではLorentz力を前提にして話をしてみたい。

 $\bm I=-enS\bm v$とかける\kyakuchuu{$\bm I$と$\bm v$は逆向きである}。
 よってこの導線の長さ$l$の部分には,$nSl$個自由電子が含まれることに注意して,
 各自由電子が$\bm f=-e\bm v\times\bm B$のLorentz力を受けているので,
 この部分が磁場から受ける力は,
 \begin{align*}
  F&=nSl\bm f=nSl(-e\bm v\times\bm B)\\
  {}&=l(-enS\bm v)\times\bm B=l\bm I\times\bm B
 \end{align*}

 また磁束密度$\bm B$の単位は右辺と左辺を比較して\unit{\ruby{T}{テスラ}}${}={}$\unit{N/A\cdot m}である。


 \section{磁場のGaussの法則}

 今度は電磁場の法則\ajMaru 2を定式化しよう。\ref{gauss}でやったことと全く同じである。
 しかし,単磁荷が発見されてないので磁場のわき出し吸いこみはないことが電場のGaussの法則と大きく異なる。よって,
 $$\int_S\bm B\cdot \bm n\,dS=\sum_{S\text 内}M=0$$
 と表せる。

 ここで単位を確認しよう。磁荷の単位は\unit{\ruby{Wb}{\text{ウェーバー}}}
 \kyakuchuu{本来なら\unit{C}と並ぶ重要な電磁気学の単位であるが,磁荷が存在しないので\unit{C}ほど頻繁に見ないかもしれない}
 で面積は\unit{m^2}なので\unit{T}${}={}$\unit{Wb/m^2}と表せる。

 \section{電磁場の法則\protect\ajMaru{4}}

 ここでは電流の周りに生じる取り囲み磁場について表現しておく。

  \subsection{Amp\'{e}reの法則}

  定性的には%図
  $I$の向きに右ねじを回して進む向きになるような,$H$のうずまきができる。
  %図

  これを式に表すと以下のようになる。
  $$\oint_C\bm H\cdot dr=\int_S\bm i\cdot\bm n\,dS$$
  ここで$C$は任意のループで左辺は$C$ 1周に沿って単位磁荷を運ぶとき,磁場のする仕事で
  右辺は%$C$で囲まれた
  $C$をふちとする面$S$を貫く全電流である。
  ここで面$S$は$C$をふちとすれば任意で,閉曲面を2つに分けた様なものを想像して,
  例えば封のしていないビニール袋でビニール袋のふちが$C$でビニール袋が面$S$というと分かりやすいだろうか?
  ただし特に大学入試の問題では面を複雑に取ると計算が難しくなるだけなので
  ちょうどシャボン玉のようにループにくっついた面を考えるのがよい。

  しかしここで疑問が生じなければおかしい。なぜなら左辺は$C$ 1周に沿って単位磁荷を運ぶとき,磁場のする仕事と書いたが
  回る方向によって正負が変わってしまう。右辺も単位法線ベクトル$\bm n$は方向を考えると2つ存在するので
  どちらを取るかで正負が変わる。なのでここは数学で頻繁に使われる右手系を使って,$\bm n$の方向に右手の親指を立てて
  他の指の向く方向が$dr$の正方向である。
  もちろん$dr$から先に決めてもいいし,右ねじを$dr$方向に回したときに進む方向を$\bm n$方向にするとしても同じことである。

  \subsection{無限に長い直線電流の周りの磁場}

  %\begin{wrapfigure}{l}{7zw}
  %\input{hi}
   %\end{wrapfigure}
   対称性より直線状どこから見ても,同じ磁場が等方的に取り囲む。よって磁場は円になりループ$C$を半径$r$の円に取ると,
   磁場$H(r)$は$H(r)\cdot 2\pi r=+I\Longleftrightarrow H(r)=\bunsuu{I}{2\pi r}$

  \subsection{無限に長いソレノイドコイルの電流の周りの磁場}

  実際のソレノイドコイルは一本の導線を巻いていくので軸に平行な電流もあるはずだがここでは理想化し,無視する。

  まず当然だが磁場にも重ね合わせの原理は成り立つので円電流が作る磁場を重ねると

  ソレノイド途中放棄


  \section{Biot-Savart(ビオ・サバール)の法則}

  Ampére の法則は積分法則なので任意の点の磁場を求められるのはきわめて対称性ときに限られる。

  よってそうでないときは電流を細かくした電流素片\kyakuchuu{実際には細かくしたら電流は流れなくなるので仮想的なもの}
  が任意の点に作る磁場の法則(Biot-Savart の法則)と重ね合わせの原理を用いる。

  Biot-Savartの法則はもともとは実験的に発見されたが,Ampéreの法則を書き換えたものと考えて欲しい。

  電流素片$I\,ds$が,この素片から任意の位置ベクトル$\bm r$の位置に作る磁場は
  $$dH=\bunsuu{1}{4\pi}\,\bunsuu{I\,ds}{|\bm r|^2}\times \bunsuu{\bm r}{|\bm r|}$$

 \section{電磁誘導}

 電磁誘導には次の2種類ある。
 \begin{enumerate}
  \item 磁場中を運動する導体中の自由電子に働くLorentz力の成分による誘導起電力
  \item 磁場の変化の周りに生じる取り囲み電場(誘導電場)による誘導起電力(=電磁場の法則\ajMaru 3)
 \end{enumerate}
 \ajMaru 1についてPQ中に考えた単位正電荷は$\mathrm{Q\to P}$向きに力$|(\bm v\times\bm B)_{\mathrm{PQ}}|$%
 ($\bm v\times\bm B$のPQ方向成分の絶対値という意味で用いたが,数学的に正しい記号ではない)
 を受ける。よって$\mathrm{Q\to P}$へ動くとき,仕事$l|(\bm v\times\bm B)_{\mathrm{PQ}}|$をされることになり,
 これが$\mathrm{Q\to P}$向きに生じる誘導起電力である。

 これによりPQ方向に電荷分布の偏りができ,$\mathrm{Q\to P}$向きの静電場が生じる。

 定常状態での力のつりあい$0=|(\bm v\times\bm B)_{\mathrm{PQ}}|-|\bm E|$

 よってQに対するPの電位は,$\phi_\text{P}-\phi_\text{Q}=+|\bm E|l=l|(\bm v\times\bm B)_{\mathrm{PQ}}|=V$

 つまり大事なのは\textgt{\bfseries 電位差と起電力は全く違う}ということである。

 (ここでは電磁誘導により)電荷に対し仕事をする状態が生じる→電荷分布が生じる→電位差が生じるというプロセスを踏むのである。
 起電力とは単位電荷に対して仕事をする能力で,電位差は考えている部分に生じた静電場から定義されるものである。

 ここは大事なので具体例をあげて違いを説明しよう。
 %%図

 \begin{description}
  \item[電位差がある回路(ここではコンデンサー)]電荷が移動して電位差を解消しようとし,解消すると電荷は移動しない
  \item[起電力がある回路(ここでは電池で考える)]電荷が移動して電位差を解消しようとするが,%
	     {\bfseries 起電力が電荷を移動させて電場を一定に保つので電位差は変わらない}
 \end{description}

  \subsection{ローレンツ変換}

  ここでまたLorentz力について考えてみよう。$\bm F=q(\bm E+\bm v\times\bm B)$という式をよく見て欲しい。
  速度$\bm v$は見る座標系によって違う。
  極端なことを言えば静止座標系で見たとき速度$\bm v$で動いてる物体は速度$\bm v$で動いている座標系から見ると
  静止しているように見える。

  よって,例えば一様磁場の中に(電場は無いとする)電荷粒子を放り込めば$q\bm v\times\bm B$を向心力とした
  等速円運動が観測されるがこの現象を粒子と同じ速度で動く座標系から見ると粒子は静止して見える。
  また,この座標系は加速度をもっているので慣性力(遠心力)が働くはずである。しかしLorentz力は\textbf 0だから
  力はつりあっていないはずである。

  現在の物理学ではローレンツ変換といい電場と磁場は見る座標系によって何らかの関係性があるとされている。
  なので,前の例の遠心力はローレンツ変換により現れる誘導電場から働く力とつり合う。

  よって電磁誘導\ajMaru 2は図のようにあくまで座標系の違いであり電磁誘導\ajMaru 1と全く同じ現象である。

  \section{電磁場の法則\protect\ajMaru{3}}

  磁場の変化の周りに電場が取り囲む。

  これを式に表すと以下のようになる。
  $$\oint_C\bm E\cdot dr=-\bunsuu{d}{dt}\int_S\bm B\cdot\bm n\,dS$$
  ここで$C$は任意のループで,左辺は$C$\,1周に生じる誘導起電力で,
  右辺は$C$をふちとする面$S$を貫く全磁束である。

  もちろん$dr,\bm{n}$の正方向は電磁場の法則\ajMaru{4}と同じである。

  また普通全磁束を$\Phi$で表し$V=-\bunsuu{d\Phi}{dt}$と書く。これをFaradayの法則という。
  またこのマイナスの符号は次のように解釈できる(レンツの法則)。

  $\Phi$\textgt{の変化を妨げる磁場をつくる電流を流そうとする向き}

  \section{自己・相互誘導}

  Amp\'ereの法則(これを変形したBiot-Savartの法則でもよい)から,電流の作る磁場は電流に比例するので
  $K_1, K_2$を(ふちとする)貫く全磁束は,%
  $%
  \begin{cases}
   \Phi_1=L_{11}I_1+L_{12}I_2\\
   \Phi_2=L_{21}I_1+L_{22}I_2
  \end{cases}$%
  とかける\kyakuchuu{コイルがいくつあっても同じ}。

  $|L_{ij}|$は$K_1, K_2$の形,配置,物質分布により決まる定数。

  このとき $|L_{11}|=L_1$を$K_1$の自己インダクタンスという。

  \hphantom{このとき} $|L_{22}|=L_2$を$K_2$の自己インダクタンスという。

  \hphantom{このとき} $|L_{12}|=|L_{21}|=M$\kyakuchuu{相反定理 証明は難}を$K_1,K_2$の相互インダクタンスという。よって%
  $\begin{cases}
    \Phi_1=+L_{1}I_1+MI_2\\
    \Phi_2=+MI_1+L_{2}I_2
   \end{cases}$%
   とかけることがわかる。ただしインダクタンスの前の符号は面の
   法線ベクトルと磁束の方向により変わりうる。ただし法線ベクトルの方向は任意に決められ,磁束の正方向も任意に決められる\kyakuchuu{嘘??}。
   %記憶が曖昧・・・・・・

   $I_1,I_2$の変化に伴い$K_1,K_2$に生じる誘導起電力は
   $\begin{cases}
     V_1=-\bunsuu{d\Phi_1}{dt}=-L_{1}\bunsuu{dI_1}{dt}-M\bunsuu{dI_2}{dt}\\[4pt]
     V_2=-\bunsuu{d\Phi_2}{dt}=-M\bunsuu{dI_1}{dt}-L_{2}\bunsuu{dI_2}{dt}
    \end{cases}$
    となる。

    またインダクタンスの単位は\unit{Wb/A}${}={}$\unit{\ruby{H}{\text{ヘンリー}}}である。

    またインダクタンスを求めるのは困難なことが多いため高校範囲で求めることが要求されるのはソレノイドコイルのみである。
    %わあああ

  \subsection{(無限に長いと近似できる)ソレノイドコイルのインダクタンス}


  \subsection{自己インダクタンス$L$のコイルの回路における役割}

  $I$と同じ向きを正とした誘導起電力$V$は$V=-L\bunsuu{dI}{dt}$と表せる。たこれは\textgt{起電力}なので
  回路方程式の左辺に書くのが普通である。また,電流$I$の変化を妨げる向きに誘導起電力が生じるので\textgt{\bfseries 逆起電力}
  とも呼ばれる。

  %\subsubsection{コイルに蓄えられる磁気エネルギー}

\chapter{Maxwell方程式のまとめ}
\begin{Enumerate}
\item $\dint{S}{}\bm E \cdot \bm n\,dS=\sum\bunsuu{Q}{\epsilon_0}$
\item $\dint{S}{}\bm B \cdot \bm n\,dS=\sum M=0$
\end{Enumerate}
ただし\ajMaru 1,\ajMaru 2の$S$は任意の閉曲面で以下の$S$は$C$を取り囲む面で閉曲面を2つに分割したようなものを想像すればよい。
\begin{Enumerate*}
\item $\displaystyle\oint_C\bm E \,dr=-\bunsuu{d}{dt}\int_S\bm B\cdot \bm n\,dS$
\item $\displaystyle\oint_C\bm H \,dr=\int_S \bm i\cdot \bm n\,dS+\bunsuu{d}{dt}\int_S\bm D\cdot \bm n\,dS$%何らかの色分けを
\end{Enumerate*}
\ajMaru 4の強調部分?は物理学では珍しくMaxwellが理論的に導入した\kyakuchuu{物理学の公式はほとんど実験的に判明したものである}。

導入した理由は\ajMaru 3の式との兼ね合いもあるがそれに加えて以下のような考察から導入しなければならないと分かった。

ループCにAmp\'ereの法則を用いると面$S_1,S_2$を考えて

\begin{align}
\oint_C\bm H \,dr&=\int_{S_1} \bm i\cdot \bm n\,dS=+I\label{amp}\\
{}&=\int_{S_2} \bm i\cdot \bm n\,dS=0?\label{max}
\end{align}

式~\eqref{max}では面$S_2$を貫く電流は存在しないので計算ミスではない。しかしよく見ると面$S_2$にはコンデンサー間の
電場が貫いている。しかも電流によりコンデンサー上の電荷が増えている。我々はこれまでの勉強により電気と磁気
は非常に密接な関係があることがわかっている。
よって\ajMaru 4の式に\ajMaru 3と同じような式が出てこないことに違和感を持つべきである%
\kyakuchuu{残念ながら磁荷は無いので“磁流”はない。したがって\ajMaru 3の式にAmp\'ereのような式は現れない}。
するとAmp\'ereの式では不十分であることが分かるはずである。

よって\ajMaru 3の式からAmp\'ereの式には$\displaystyle\bunsuu{d}{dt}\int_S\bm D\cdot \bm n\,dS$の項が必要である。

符号は実際に面$S_2$を貫く電束密度を求めてみればよい\kyakuchuu{変位電流ともいう}。

電荷$Q$,面積$S$とすればコンデンサー間の電束密度は$\bm D=\bunsuu{Q}{S}$で$I=\bunsuu{dQ}{dt}$より

$\bunsuu{d}{dt}\dint{S}{}\bm D\cdot \bm n\,dS=\bunsuu{dQ}{dt}=+I$なので符号は正であれば面$S_1$と結果は同じになる。

もちろんこの議論だけでは予想に過ぎず実際にこの項が現れることは実験をして示さねばならない。
そして現在ではこの項が必要であることは実験的に示されている。

そして以上の議論から次の現象が予言される。

導線に振動電流を流す。すると振動電流の周りには\ajMaru 4により振動磁場が生じ,その周りに\ajMaru 3により振動電場
が生じ,…

というようにして振動電磁場が空間を伝える現象があることになる。これが電磁波と呼ばれるもので
Maxwellは電磁波の伝わる速度を理論的に計算し光と同じ速度であることを突き止め,光も電磁波であることを証明した。

\end{document}
