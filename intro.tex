\section{物理の勉強の仕方}

基本原理に基づいて,その原理を理解し(人に教授することのできるレベル),種々の演習を通じて
確認する。自然なものの見方が身につけばもうそれで終わり。

物理において『東大物理』も『医系物理』もなく,『古典物理学』『現代物理学』を合わせて『物理学』
と総称する。また物理において解法などという言葉は使用しない。なぜなら,\textgt{\bfseries 自然なものの見方が身に
ついた人からみれば,その方針が自然に出てくるから}である。

我々が日本語を話す時にいちいち考えて話さないのと同様に,
物理においてもそういった自然に理解できるような見方を身につけることが重要。
\subsection{古典物理学(19世紀までの物理学)の世界観}

古典物理学の世界観は“素朴な”素粒子論\kyakuchuu{現代物理学においては素粒子論という発想が古典物理学とは異なる}
であり,これが言わんとしていることは
「生命だろうが,心(スピリチュアルな世界)だろうが,
すべては究極粒子の振る舞いによって表現される」
ということ。

例えば,我々が考えていることは脳内の素粒子の振る舞い(運動)によって決定されるといったこと。

ただし,今までの内容はあくまでも前提(人間のアイデアの1つ)にすぎず,こういった前提をすることによって
「つじつまが合いそうだ」という発想から生まれたにすぎない。だから,『世界観』という表現を使用する。
学問をする上で,大切な事は「その学問が何を前提としているのか」や「どういった発想を基に,
何を目標としているのか」をとらえることである。当然そこには善し悪しといったレベルの議論は存在
しない(どれも1つの世界観に過ず,絶対的なものではない)。
\subsection{自然科学について}

自然科学は自然界を見るときの基本的な世界観でつじつまが合わなければもちろん修正される
\kyakuchuu{事実,古典力学的な世界観は19世紀ごろに修正された%
(ただし現象をマクロに見るときには依然として古典力学は有用である)}。

絶対的なものではないが,実際問題として上手くいっているので,使っているだけである。
%“素朴な”素粒子論は単純明快
すべての自然現象は素粒子の振る舞いとしてとらえられる。
その真偽判定は実証(現実との照らし合わせ)のみによる。
よって,有限な例についてしか,示すことが出来ない!
(何度もいうようだが,絶対的ではない。)

しかし物理の表現を文学的な言葉で行うと(例えば日本語)分からない人が出てくるし,
本来同じ概念として捉えられるべき概念が人によって異なったものに受け取られる可能性がある
逆に言えば,文学的な言葉はそのあいまいさ故に,多種多様な解釈を可能にし,人類の精神活動を豊かな
ものにするのに重要な役割をしたともいえる。しかしそれだと議論にならないので,
表現は定義ができるだけ正確なものを用いなければならない。

それを解決するために人間が作った言語。それが“数学”である。

これで『なぜ数学は必要か』という質問の答えは明らかになった。
万人が納得するような言語が必要だったからである。

だから自然科学の世界においては,その世界の言葉である“数学”ができないと大きなハンデを
負うことになるのである。
